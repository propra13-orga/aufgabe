\documentclass{programmierpraktikum}
\vorlesung{Programmierpraktikum}
\semester{Sommersemester 2013}
\betreuer{Wilfried Linder}
\subtitle{Dungeon Crawler --- Meilenstein 2}

\begin{document}

\maketitle
Im zweiten Meilenstein soll unser \textbf{Dungeon Crawler} etwas aufregender gemacht werden.
Dazu werden eine Reihe weiterer fester Anforderungen kommen. Eine gute Softwarearchitektur kann
dafür sorgen, dass diese Anforderungen problemlos umgesetzt werden können.
%
\section{Anforderungen für den zweiten Meilenstein}

\subsection{Levelstruktur}
Im ersten Meilenstein sollten 3 Räume erstellt werden. Diese 3 Räume oder mehr bilden nun den
ersten Level. Im letzten Raum soll es einen Bossgegner geben. Wenn dieser besiegt ist geht es
ins nächste Level. Wieviele Level sie implementieren bleibt Ihrer Fantasie überlassen.
Mindestens jedoch 3 mit jeweils mindestens 3 Räumen.

Der letzte Bossgegner muss "`besonders schwer"' sein.
\subsection{Räume aus Textdateien einlesen}
Anstatt die Räume fest im Code als Arrays oder ähnliches zu kodieren, sollten diese aus
Textdateien eingelesen werden können. Die Textdatei sollte von Menschen les- und bearbeitbar
sein. Ein Beispiel sehen sie in Listing.~\ref{lst:txtexample}.
%
\begin{center}
  \begin{listing}[h!]
    \begin{minted}[gobble=6,bgcolor=mintedbg]{text}
      ################
      #  F           G
      # # # # # # #  #
      # # # # # # #  #
      # # # # # # #  #
      # # #F# # # #  #
      # # # # # # #  #
      # # # # # # #  #
      # # # # # # #  #
      # # # # # # #  #
      S        F     #
      ################
    \end{minted}
    \caption{Beispiel-Raum-Datei. Die Rauten könnten Wände, das S den Eingang, das G den Ausgang und
    die Fs irgendwelche Gegner sein. Welche Zeichen Sie verwenden bleibt Ihnen
    überlassen.\label{lst:txtexample}}
  \end{listing}
\end{center}
%
\subsection{Checkpoints}
Nach dem Sterben wieder von vorne beginnen zu müssen, kann frustrierend sein. Gönnen Sie dem Spieler
daher die Möglichkeit an bestimmten festgelegten Stellen (z.B. jeder Eingang eines Raumes) wieder
aufzutauchen, wenn der Spieler gestorben ist.  Allerdings darf das nicht unbegrenzt vonstatten gehen.
Begrenzen Sie es durch eine bestimmte Anzahl an Leben.
%
\subsection{Bewegliche Gegner}
Hauchen Sie den Gegnern mehr Leben ein und ermöglichen Sie dem Spieler diesen Gegnern das Leben
wieder auszuhauchen.

Es darf aber trotzdem (zusätzlich) bewegliche Fallen geben (z.B. Feuerbälle), die
unzerstörbar sind. Ihrer Fantasie sind keine Grenzen gesetzt.
%
\subsection{Waffen}
Gegner und Spieler sollen sich nicht einfach nur berühren müssen, um sich zu schaden, sondern auch
über Waffen (Nah- oder Fernkampf) verfügen.
%
\subsection{Zauber}
Zusätzlich zu Waffen sollen Spieler und Gegner auch zaubern können. Dazu braucht es Mana, das
durch das Zaubern verbraucht wird.
%
\subsection{Schadenssystem}
Um das Spiel etwas moderner zu gestalten sollen Spieler und Gegner auch Schaden, der über Tod oder
Leben hinausgeht, erleiden können. Jede Waffe/jeder Zauber soll eine bestimmte Menge an Schaden
austeilen, die der Spieler/Gegner erleidet und daraufhin Lebenspunkte verliert. Gehen diese Lebenspunkte
auf 0, ist der Spieler tot und verliert ein Leben, bzw. ist der Gegner einfach tot.

Da niemand gerne nackt rumläuft soll der Spieler auch noch über Rüstung verfügen, die den Schadenswert mindert.
Sämtliche Berechnungsfunktionen, die dafür erforderlich sind, sowie das Maximum der Lebenspunkte usw.
können Sie sich selbst ausdenken.
%
\subsection{Items}
In der Welt sollen Items verteilt sein, die dem Spieler helfen. Zum Beispiel Pakete, die die Lebenspunkte
auffüllen (Healthpacks), Manatränke, Waffen, Munition und was Ihnen noch einfällt. Mindestens müssen
Healthpacks, Manatränke und Waffen sowie Geld (s.u.) implementiert werden.
%
\subsection{Shopsystem}
Als besonderes Item soll es Geld geben, mit dem man an Shops, die in der Welt verteilt sind entweder alle
anderen, oder auch zusätzliche, oder eben nicht alle anderen sondern nur einen Teil aller
Items kaufen kann. Den Modus können Sie entscheiden.
%
\subsection{NPCs zur Interaktion}
Außer dem Shop soll es noch mindestens einen weiteren NPC geben, der in irgendeiner Weise mit dem
Spieler interagiert (z.B. die Story erzählt).
%
\subsection{Informationsleiste}
Alle relevanten Informationen (Lebenspunkte, Manapunkte, aktuelle Waffe, etc.) müssen irgendwie auf
dem Bildschirm dargestellt werden. In sehr vielen Spielen gibt es beispielsweise im unteren Bereich eine rote Kugel
für Lebenspunkte, eine blaue Kugel für Manapunkte und dazwischen steht der Rest.
%
\section{Abgabe}
Die Abgabe erfolgt wie beim ersten Meilenstein, wie mit Ihrem Tutor abgesprochen.

Stichtag für den zweiten Meilenstein ist der \textbf{Übungstermin in der Woche vom 17. -- 22. Juni 2013}
\end{document}
